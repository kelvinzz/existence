\documentclass[a4paper, 11pt]{amsart}
%\documentclass{article}
\usepackage{a4wide}
\usepackage{latexsym, fullpage}
\usepackage[left=2cm,top=3cm,right=2cm]{geometry}
\usepackage[utf8]{inputenc}
\usepackage{amsmath}
\usepackage{amsfonts}
\usepackage{amsthm}
\usepackage{amssymb}
\usepackage{amscd}
\usepackage[all]{xy}
\usepackage{fancyhdr}
\usepackage{enumerate}
\usepackage{accents}
\usepackage{setspace}
\usepackage{xcolor}
\usepackage{upgreek}
\usepackage{verbatim}
\usepackage[mathscr]{eucal}
\usepackage{mathrsfs}
\usepackage[numbers]{natbib}
\usepackage[fit]{truncate}
\usepackage{graphicx}
\usepackage{epsfig}
\usepackage{soul}
%\usepackage{pdfsync}
%\usepackage{authblk}
%\usepackage{refcheck}

%\addtolength{\topmargin}{-1mm}
%\addtolength{\textheight}{3mm}

%\newcommand{\truncateit}[1]{\truncate{0.8\textwidth}{#1}}
%\newcommand{\scititle}[1]{\title[\truncateit{#1}]{#1}}
\newcommand{\kel}[1]{\textbf{\textcolor{purple}{#1}}} 

\usepackage{tabularx,ragged2e,booktabs,caption}
\newcolumntype{C}[1]{>{\Centering}m{#1}}
\renewcommand\tabularxcolumn[1]{C{#1}}

\numberwithin{equation}{section}

% newtheorem

\theoremstyle{plain}
\newtheorem{theorem}{Theorem}[section]
\newtheorem{corollary}[theorem]{Corollary}
\newtheorem{lemma}[theorem]{Lemma}
\newtheorem{proposition}[theorem]{Proposition}
\newtheorem{claim}[theorem]{Claim}
\newtheorem{remark}[theorem]{Remark}
\newtheorem{assumption}{Assumption}
%\newtheorem{proof}[theorem]{Proof}

\newtheorem{question}[theorem]{Question}
\newtheorem{conjecture}[theorem]{Conjecture}

\theoremstyle{definition}
\newtheorem{definition}[theorem]{Definition}
%\newtheorem{remark}[theorem]{Remark}
\newtheorem{example}[theorem]{Example}

\theoremstyle{remark}
\newtheorem*{notation}{Notation}
\newtheorem{exercise}[theorem]{Exercise}

% Common mathematical symbols
\newcommand{\CC}{\mathbb{C}}
\newcommand{\RR}{\mathbb{R}}
\newcommand{\R}{\mathbf{R}}
\newcommand{\QQ}{\mathbb{Q}}
\newcommand{\ZZ}{\mathbb{Z}}
\newcommand{\NN}{\mathbb{N}}
\newcommand{\N}{\mathbf{N}}
\newcommand{\HH}{\mathbb{H}}
\newcommand\dom{{\mathop{\rm dom}}}
\newcommand\Dom{{\mathop{\rm Dom}}}
\newcommand{\Lip}{\operatorname{Lip}}

\newcommand{\argmin}{\operatornamewithlimits{argmin}}

\makeatletter
\def\munderbar#1{\underline{\sbox\tw@{$#1$}\dp\tw@\z@\box\tw@}}
\makeatother


\allowdisplaybreaks
%\renewcommand\Authands{ and }

%\doublespacing

%\date{\today}


\title{Existence in Multidimensional Screening with General Nonlinear Preferences}%without Quasilinearity}%{IMPLEMENTABILITY WITHOUT QUASILINEARITY}%\thanks{}

\thanks{
	$^*$ The author would like to express his deepest gratitude to his Ph.D. advisor Robert J. McCann for leading him to this project, and for his guidance and inspiration throughout. The author is grateful to Xianwen Shi, Alfred Galichon, Guillaume Carlier and  Ivar Ekeland  for stimulating conversations and encouragement,
	as well as to Georg N\" oldeke and Larry Samuelson for sharing their work in preprint form and vital remarks. 
	This project was initiated during the Fall of 2013 when the author was in residence at the Mathematical Sciences Research Institute in Berkeley CA, under a program supported by National Science Foundation Grant No. 0932078 000, and progressed during the Fall 2014 program of the Fields Institute for the Mathematical Sciences.
	\copyright \today}




\author{Kelvin Shuangjian Zhang$^\dagger$}\thanks{$^\dagger$Department of Mathematics, University of Toronto, Toronto, Ontario, Canada, M5S 2E4 {\tt szhang@math.toronto.edu}}



\begin{document}

\begin{abstract}
	We generalize the approach of Carlier (2001) and provide an existence proof for the multidimensional screening problem with general nonlinear preferences. 
	%A monopolist wishes to maximize her profit by finding an optimal price menu. Given the price list by monopolist, each agent will choose to buy that product which maximizes his own utility. Then the principal will maximize her total profits which is the sum of net earnings from each product sold, whose distribution is based on the choices of agents, and thus fundamentally depends on the distribution of agents and also price menu. In this paper, we provide an existence proof of adverse-selection principal-agent problems, a generalization Carlier's approach (2001), without requiring quasilinear utility. 
	%Using $G$-convex analysis, we 
	{We first} formulate the principal's {problem} %objective 
	as a maximization problem with $G$-convexity constraints, %before presenting the existence result.
	{and then use $G$-convex analysis to prove existence.} \medskip 

	{\it Keywords:} Principal-agent problem; Adverse selection; Bi-level optimization; Incentive-compatibility; Non-quasilinearity
\end{abstract}

\bigskip

\maketitle

\section{Introduction}\label{section:introduction}
	This paper provides a general existence for a multidimensional nonlinear pricing model,  which is a natural extension of the models studied by Mussa-Rosen \cite{MussaRosen78}, %Roberts \cite{Roberts79}, 
	Spence \cite{Spence74, Spence80}, Myerson \cite{Myerson81}, Baron-Myerson \cite{BaronMyerson82}, Maskin-Riley \cite{MaskinRiley84}, Wilson \cite{Wilson93}, Rochet-Chon$\acute{e}$ \cite{RochetChone98}, Monteiro-Page \cite{MonteiroPage98} and  Carlier~\cite{Carlier01}. A major distinction lies in whether the private type is one-dimensional (such as \cite{MussaRosen78, MaskinRiley84}), or multidimensional (such as \cite{RochetChone98,MonteiroPage98, Carlier01}). Another distinction is whether preferences are quasilinear on price (such as \cite{Armstrong96, Carlier01}) or fully nonlinear (such as \cite{NoldekeSamuelson15p, McCannZhang17}), especially for multidimensional models.  
	\medskip
	
	This paper proves the existence of a (price menu) solution to a multidimensional multiproduct monopolist problem, {by extending} %in an extension of 
	Carlier \cite{Carlier01} to fully nonlinear preferences. $G$-convex analysis is employed to deal with the difficulty of non-quasilinear preferences. This method is potentially applicable to other problems under the same principal-agent framework, such as the study of tax policy (\cite{Mirrlees71}) and other regulatory policies (\cite{BaronMyerson82}). \medskip

%	Most of the papers you listed are about monopolistic screening;  \medskip
	


	Consider the problem for a multiproduct monopolist who sells indivisible products to a population of consumers, who each buy at most one unit. Assume there is neither cooperation, nor competition between agents. Additionally, assume the monopolist is able to produce enough of each product such that there are neither product supply shortages {nor economies of scale}. Taking into account participation constraints and incentive compatibility, the monopolist would like to find the optimal menu of prices to maximize its total profit.\medskip

	In this paper, we first identify incentive compatibility with a $G$-convexity constraint, before rewriting the maximization problem by converting the optimization variables from a product-price pair of mappings to a product-value pair. It can then be shown that the product-value pair converges under the $G$-convexity constraint. The existence result follows. \medskip






	
Starting from Mirrlees \cite{Mirrlees71} and Spence \cite{Spence74}, there are two main types of generalizations. One generalization is in terms of dimension, from 1-dimensional to multi-dimensional. The other generalization is in utility functional form, from quasilinear to non-quasilinear.\medskip



For the quasilinear case, where utility depends linearly on price, theories of existence \cite{Basov05,RochetStole03,Carlier01,MonteiroPage98}, uniqueness \cite{CarlierLachand-Robert01,MussaRosen78,RochetChone98} and robustness \cite{BaronMyerson82,FigalliKimMcCann11} have been widely studied, among which the equivalence of function space convexity to the non-negative cross-curvature condition revealed by Figalli-Kim-McCann \cite{FigalliKimMcCann11} serves as a major milestone {orienting our work}. \medskip


When parameterization of preferences is linear in agent types and price, Rochet and Chon$\acute{e}$ (1998, \cite{RochetChone98}) not only obtain existence results but also partially characterize optimal solutions and expound their economic interpretations, %the solution for optimality with economic interpretations, 
given that monopolist profits can be characterized by {the aggregate difference between selling prices and manufacturing costs.}\medskip

More generally, Carlier \cite{Carlier01}) proves existence results for general quasilinear utility, where agent type and product type are not necessarily of the same dimension and monopolist profit equals selling price minus some linear manufacturing cost.\medskip

This paper generalizes the quasilinear case to the non-quasilinear case, which has many applications. For example, some fully nonlinear utilities include scenarios where agents are more sensitive to higher prices and where different agents might have different sensitivities to the same price. %utility $G(x,y,p(y)) = b(x,y) - p^3(y)$ describes the scenario when agents are more sensitive to the higher price, given all prices are greater than 1; and $G(x,y,p(y))=b(x,y)-f(x,p(y))$ presents one of the situations where various agents would have different sensitivities to the same price. 
See Wilson~\cite[Chapter 7]{Wilson93} for the importance of taking income effects into account. 
The generalized existence problem is also mentioned as a conjecture by Basov \cite[Chapter~8]{Basov05}. However, only a few results are known for the multidimensional non-quasilinear case, and the impact of price on utility could be much more complicated.\medskip


Recently, N$\ddot{o}$ldeke and Samuelson (2015, \cite{NoldekeSamuelson15p}) provide a general existence result given that the consumer and product space are compact, by implementing a duality argument based on Galois Connections. McCann-Zhang \cite{McCannZhang17} show another general existence result given that the consumer and product space are bounded, using $G$-convexity, and generalize uniqueness and convexity results of Figalli-Kim-McCann \cite{FigalliKimMcCann11} to the non-quasilinear case. In this paper, we also explore existence using $G$-convex analysis, which will be introduced in Subsection \ref{subsection:preliminary}, but with less restriction on boundedness of the product domain and without assuming the generalized single-crossing condition. As a result of lack of natural compactness, the proof of this paper is quite different from that of either of the %se 
 earlier papers. It should be mentioned here that {neither} this paper, %, together with these two papers
{nor the earlier two}, %do not 
require the monopolist profit to take on a special form, which is a generalization from much of the literature.\medskip


 
%In our paper, we will present another general existence result for the principal-agent framework where the utility function is decreasing with respect to its third variable but on more relaxed domains, where $G$-convexity plays a crucial role, which distinguishes it from N$\ddot{o}$ldeke-Samuelson's work \cite{NoldekeSamuelson15p}. 



The remainder of this paper is organized as follows. Section \ref{section:model} states the mathematical model and assumptions. It also introduces preliminaries, including $G$-convexity, $G$-subdifferentiability, and reformulation of the monopolist's problem. In Section \ref{section:mainresult}, we state the existence theorem as well as the convergence proposition. Section \ref{section:futurework} proposes some directions of future work. We leave all the proofs of lemmas, propositions and the existence theorem in Section \ref{section:proofs}. 
 

\bigskip
 

\section{Model}\label{section:model}



{This model is a bilevel optimization. After a monopolist publishes its price menu, each agent maximizes his utility through the purchase of at most one product. The monopolist maximizes aggregate profits based on agents' choices, knowing only the statistical distribution of agent types.}\medskip


Suppose agents' preferences are given by some parametrized utility function  $G(x, y, z)$, where $x$ is a $M$-dimensional vector of consumer characteristics, $y$ is a $N$-dimensional vector of attributes of each product, and $z$ represents the price of each product. Denote {by} $X$ the space of agent types, by $Y$ the space of products, by $cl(Y)$ the closure of $Y$, by $Z$ the space of prices, and by $cl(Z)$ the closure of $Z$. {In this paper, we only consider the case where both agent types and product attributes are continuous.} \medskip

The monopolist sells indivisible products to agents. Each agent buys at most one unit of product. We assume no competition, cooperation or communication between agents. For any given price menu $p: cl(Y) \longrightarrow cl(Z)$, agent $x \in X$ knows his utility $G(x,y,p(y))$ for purchasing each product $y$ at price $p(y)$. It follows that every agent solves the following maximization problem 
\begin{equation}\label{eqn_optimal_product}
	u(x):=\max_{y \in cl(Y)} G(x, y, p(y)),
\end{equation}
where $u(x)$ represents the maximal utility agent $x$ can obtain, and $u: X \longrightarrow \R$ is also called the value function or indirect utility function.%, given utility function $G: X \times cl(Y) \times Z \longrightarrow \R$.{\marginpar \color{blue} Jiaqi: - is this math necessary? - economists understand utility much better as a f(x,y,p(y)) — i.e. x,y,p(y) are inputs and plugging them into the function gives utility (#) output.. doesn’t need to be defined as  a mapping (which is more technical..) - generally, don’t need to define utility, profits, and cost functions as mappings (because they’re standard)—> is the mapping important to the math proof of your paper?}
~At this point, it is assumed that the maximum in \eqref{eqn_optimal_product} is attained for each agent $x$.
\medskip

If agent $x$ purchases product $y$ at price $p(y)$, the monopolist would earn from this transaction a profit of $\pi(x,y,p(y))$%, given that $\pi: X \times cl(Y) \times Z \longrightarrow \R$ 	
.  {For example, monopolist profit can take the form $\pi(x,y,p(y)) = p(y)-c(y)$, where %the function $c: Y\longrightarrow \R$ represents manufacturing cost.} {\marginpar \color{blue} - sufficient to say that c(y) is a variable manufacturing cost function..  }
c(y) is a variable manufacturing cost function.} Summing over all agents in the distribution $d\mu(x)$, the monopolist's total profit is characterized by 
\begin{equation}\label{eqn_monopolist_integral}
	\Pi(p, y):=\int_{X} \pi(x, y(x), p(y(x))) d\mu(x),
\end{equation}
which depends on her price menu $p: cl(Y) \longrightarrow cl(Z)$ and  agents' choices $y: X \longrightarrow cl(Y)$.\footnote{It is worth mentioning that in some literature, the monopolist's objective is to design a product line $\tilde{Y}$ (i.e.~a subset of $cl(Y)$) and a price menu $\tilde{p}: \tilde{Y} \longrightarrow \R$ that jointly maximize overall monopolist profit. Then, given $\tilde{Y}$ and $\tilde{p}$, an agent of type $x$ chooses the product $y(x)$ that solves
	\begin{equation*}
		\max_{y \in \tilde{Y}} G(x,y, \tilde{p}(y)):= u(x).
	\end{equation*}
Allowing price to take value $\bar{z}$ (which may be $+\infty$), and assuming Assumption \ref{assmp:Gregular} below, the effect of designing product line $\tilde{Y}$ and price menu $\tilde{p}: \tilde{Y}\longrightarrow \R$ is equivalent to that of designing a price menu $p : cl(Y)\longrightarrow (-\infty, +\infty]$, which equals $\tilde{p}$ on $\tilde{Y}$ and maps $cl(Y) \setminus \tilde{Y}$ to $\bar{z}$, such that no agents choose to purchase any product from $cl(Y) \setminus \tilde{Y}$, which is less attractive than the outside option $y_{\emptyset}$ according to Assumption \ref{assmp:Gregular}. In this paper, we use the latter as the monopolist's objective.
 \vspace{0.1cm}
 
For any given price menu $p: cl(Y)\longrightarrow (-\infty, +\infty]$, one can construct a mapping $y: X \longrightarrow cl(Y)$ such that each $y(x)$ solves the maximization problem in \eqref{eqn_optimal_product}. But such mapping is not unique, for some fixed price menu, without the single-crossing type assumptions. %JIAQI- either use both commas, or can remove~
Therefore, we adopt in \eqref{eqn_monopolist_integral} the total profit as a functional of both price menu~$p$ and its corresponding mapping $y$.}\medskip

%where $x\in X \longmapsto y(x) \in cl(Y)$ denotes that product which agent $x$ chooses to buy, $p: cl(Y) \longrightarrow Z$ represents her price menu and $d\mu(x)$ stands for the distribution of agents.\medskip

% DISCUSS/CLARIFY WITH XIANWEN:
{%Since agents' individual characteristic is not observable by the monopolist who only knows its distribution
Since the monopolist only observes the overall distribution of agent attributes, and is unable to distinguish individual agent characteristics, the monopolist takes into account the following incentive-compatibility constraint when determining product-price pairs $(y, p(y))$.
	}

\begin{definition}[incentive compatibility]
	A (product, price) pair of measurable mappings $(y,z): X \longrightarrow cl(Y) \times cl(Z)$ on agent space $X$ is \textit{incentive compatible} if and only if $G(x,y(x),z(x)) \ge G(x, y(x'), z(x'))$ for all $(x,x')\in X^2$.
\end{definition}

Therefore, an incentive-compatible product-price pair $(y, p(y))$ ensures that no agent has incentive to pretend to be another agent type.\medskip 


In addition, we adopt a participation constraint in order to rule out the possibility of the monopolist charging exorbitant prices and the agents still having to make transactions despite this: each agent $x\in X$ will refuse to participate in the market if the maximum utility he can obtain %, $u(x)$, 
is less than %some predetermined number 
his reservation value $u_{\emptyset}(x)$, where the function $u_{\emptyset}: X \longrightarrow \R$ is given in the form $u_{\emptyset}(x): = G(x, y_{\emptyset}, z_{\emptyset})$, for some $(y_{\emptyset}, z_{\emptyset}) \in cl(Y \times Z)$. \medskip

{For monopolist profit, some literature assumes $\pi(x, y_{\emptyset}, z_{\emptyset}) \ge  0$ for all $x\in X$ to ensure that the outside option is harmless to the monopolist. Here, it is not necessary to adopt such assumption for the sake of generality. \medskip}

Then the monopolist's problem can be described as follows:


\begin{equation}\label{origin_problem}
(P_0)
\begin{cases}
\sup \Pi(p,y)=\int_{X} \pi(x, y(x), p(y(x)))~ d\mu(x)\\
s.t.\ (y,p(y)) \text{~is incentive compatible%incentive compatibility
	};\\
s.t.\  G(x, y(x), p(y(x))) \ge u_{\emptyset}(x) \text{ for all } x \in X;\\
s.t.\ p \text{  is lower semicontinuous}.
\end{cases}
\end{equation}

We assume that $p$ is lower semicontinuous, without which the maximum in \eqref{eqn_optimal_product} may not be attained. We also rewrite the monopolist's problem in Proposition \ref{equiv_form}, which is an equivalent form of \eqref{origin_problem}. However, in that equivalence form% of $(P_0)$
, the lower-semicontinuity constraint will be encoded in $G$-convexity of the value functions%, which will be shown in Proposition \ref{equiv_form}
.\medskip
%\vspace{1cm}

The purpose of the following subsections is to fix terminology and prepare the preliminaries for the main results of the next section.
The proofs can be found in Section \ref{section:proofs}.\medskip

\subsection{$G$-convex and $G$-subdifferentiability}
\label{subsection:preliminary}

In this subsection, we introduce some tools from convex analysis and the notion of $G$-convexity, which is a generalization of ordinary convexity. {Several results in this subsection echo those from Section 3 of McCann-Zhang \cite{McCannZhang17} but extended to unbounded domains.}  \medskip

The followings are assumptions to the agents' direct utility $G$. We use $C^0(X)$ to denote the space of all continuous functions on $X$, and use $C^1(X)$ to denote the space of all differentiable functions on $X$ whose derivative is continuous.  \medskip
 
 \begin{assumption}\label{assmp:Gregular}
 	Agents' utility $G \in C^{1}(cl(X\times Y \times Z))$, where the space of agents $X$ is a bounded open convex subset in $\R^M$ with $C^1$ boundary%$\R_{+}^M$ with $C^1$ boundary {which does not intersect with the boundary of $\R_{+}^M$}
 	, the space of products $Y \subset \R^N$%$Y \subset \R^N_{ +}$ needed by coordinate-monotonicity
 	, and range of prices $Z=(\munderbar z,\bar z)$ with $-\infty < \munderbar z < \bar z \le +\infty$. {Assume $G(x,y,\bar{z}) := \lim_{z\longrightarrow \bar{z}} G(x,y,z) \le G(x, y_{\emptyset}, z_{\emptyset})$, for all $(x,y) \in X \times cl(Y)$; and assume this inequality is strict when $\bar{z} = +\infty$.}
 \end{assumption}
 
 Here we do not necessarily assume $X$, $Y$, and $Z$ are compact spaces; in particular, $Y$ and $Z$ are potentially unbounded %(i.e.\ we do not set \textit{a priori} bounds for product space or price).
 (i.e.\ we do not set  \textit{a priori} bounds for product attributes or an \textit{a priori} upper bound for price). However, we do specify a lower bound for the price range, since the monopolist has no incentive to set price close to negative infinity. %{\color{purple} Here $\bar{z}$ is an uninteresting price to all the agents, for example $\bar{z} = +\infty$. }%, {as well as an upper bound for price range, since agents would purchase an outside product when price equals positive infinity.} 
 \medskip
 
 
 
 \begin{assumption}\label{assmp:Gdecreasing}
 	$G(x,y,z)$ is strictly decreasing in $z$ %{ and $G_z(x,y,z)<0$ for each $(x,y,z)\in X \times Y \times Z$}, %FOR EACH
 	for each $(x,y) \in cl(X \times Y)$.
 \end{assumption}
 
 
 This is to say that, the higher the price paid to the monopolist, the lower the utility that will be left for the agent, for any given product. %natural since agent's benefit is decreasing on the utility transferred to the monopolist, for the fixed product.
 For each $(x, y) \in X\times cl(Y)$ %AND
 and $u\in G(x,y, cl(Z))$, define $H(x,y,u) := z$ %WHENEVER
 whenever $G(x,y,z) = u$, i.e., $H(x, y, \cdot)= G^{-1}(x,y,\cdot)$. Therefore, $H(x,y,u)$ represents the price paid by agent $x$ for product $y$ when receiving value $u$.\medskip
 
 Proposition \ref{lemma_continuity} shows that the inverse function of $G$ is also continuous, because $G$ is continuous and monotonic on the price variable. \medskip
 
 \begin{proposition}\label{lemma_continuity}
 	Given Assumption \ref{assmp:Gregular} and Assumption \ref{assmp:Gdecreasing}, function $H$ is continuous.
 \end{proposition}

{Recall that the subdifferential of a (convex) function $u$ at $x_0$ is defined as the set:
	\begin{equation*}
	\partial u(x_0) = \{ y \in cl(Y)| u(x) - u(x_0) \ge \langle  x- x_0,  y \rangle, \text{ for all } x \in X  \}.
	\end{equation*}
	Here we use $ \langle , \rangle$ to denote the Euclidean inner product. This set is nonempty for every $x_0$ if and only if $u$ is convex.} Then for any convex function $u$ on $X$ and any fixed point $x_0 \in X$, there exists $y_0 \in \partial u(x_0)$, satisfying% {$u(x) - u(x_0) \ge \langle x - x_0, y_0\rangle$}, i.e., 
\begin{equation}\label{convex_function}
u(x) \ge  \langle x , y_0\rangle -( \langle x_0, y_0\rangle -  u(x_0)),	\text{  for all $x \in X$},
\end{equation} 
where equality holds at $x = x_0$. On the other hand, if for any $x_0\in X$, there exists $y_0$, such that \eqref{convex_function} holds for all $x\in X$, then $u$ is convex. The following definition is analogous to this property, which is a special case of $G$-convexity, when $G(x,y,z) = \langle x, y \rangle -z$.

\begin{definition}[$G$-convexity]
	A function $u\in C^0(X)$ is called {\it $G$-convex} if for each $x_0 \in X$, there exist $y_0 \in   cl(Y)$, and $z_0 \in  cl(Z)$ such that $u(x_0)=G(x_0, y_0, z_0)$, and $u(x)\ge G(x, y_0, z_0)$, for all $x\in X$.
\end{definition}


Similarly, one can also generalize the definition of subdifferential from \eqref{convex_function}.

\begin{definition}[$G$-subdifferentiability]
	The $G$-subdifferential of a 
	%$G$-convex % 
	function $u: X \longrightarrow \R$ is a point-to-set mapping defined by
	\begin{equation*}
	\partial^G u(x):= \{ y\in  cl(Y)| u(x')\ge G(x',y, H(x,y,u(x))), \text{ for all } x'\in X\}.
	\end{equation*}
	
	A function $u$ is said to be {\it $G$-subdifferentiable} at $x$ if and only if $\partial^G u(x) \neq \emptyset$.\footnote{In Trudinger \cite{Trudinger14}, this point-to-set mapping $\partial^G u$ is also called $G$-$normal$ mapping; see paper for more properties related to $G$-convexity.}
\end{definition}




In particular, if $G(x,y,z) = \langle x, y \rangle - z$, then the $G$-subdifferential coincides with the subdifferential. There are other generalizations of convexity and subdifferentiability. For instance, $h$-convexity in Carlier \cite{Carlier01}, or equivalently, $b$-convexity in Figalli-Kim-McCann \cite{FigalliKimMcCann11}, or $c$-convexity in  Gangbo-McCann \cite{GangboMcCann96}, is a special form of $G$-convexity, where $G(x,y,z)=  h(x,y) -z$, which serves an important role in the quasilinear case. For more references of convexity generalizations, see Kutateladze-Rubinov \cite{KutateladzeRubinov72}, Elster-Nehse \cite{ElsterNehse74}, Balder \cite{Balder77}, Dolecki-Kurcyusz \cite{DoleckiKurcyusz78},  Singer \cite{Singer97},  Rubinov \cite{Rubinov00a}, and Martínez-Legaz \cite{MartinezLegaz05}.\medskip


As mentioned above, it is known in convex analysis that a function is convex if and only if it is subdifferentiable everywhere. The following lemma adapts this to the notion of $G$-convexity. 

\begin{lemma}\label{convex-subdiff}
	Given Assumption \ref{assmp:Gdecreasing}, a function $u: X \longrightarrow \R$ is $G$-convex if and only if it is $G$-subdifferentiable everywhere.
\end{lemma}

Using Lemma \ref{convex-subdiff}, one can show the following result, which identifies incentive compatibility using $G$-convexity and $G$-subdifferentiability.

\begin{lemma}\label{incen/convex}
	Let $(y,z)$ be a pair of mappings from $X$ to $cl(Y) \times cl(Z)$. Given Assumption \ref{assmp:Gdecreasing}, this (product, price) pair is incentive compatible %incentive-compatible {\marginpar{Jiaqi: \color{blue} - don’t need hyphen when it’s not being used as an adjective 	(e.g. ‘incentive-compatibility constraint’)}}
	if and only if $u(\cdot):=G(\cdot,y(\cdot),z(\cdot))$ is $G$-convex and $y(x)\in \partial^G u(x)$ for each $x \in X$.
\end{lemma}


\subsection{Implementability}\label{subsection:implementability}
{We introduce implementability here, which is closely related to incentive-compatibility and can also be exhibited by $G$-convexity and $G$-subdifferential. %Note that in the following statements, no assumptions of single-crossing type are required.
	
	\begin{definition}[implementability]
		A function $y: X \longrightarrow cl(Y)$ is called \textit{implementable} if and only if there exists a function $z: X \longrightarrow \R$  such that the pair $(y, z)$ is incentive compatible.
	\end{definition}
	
	\begin{remark}\label{rmk:implementability}
		Allowing Assumption \ref{assmp:Gdecreasing}, a function $y$ is implementable if and only if there exists a price menu $p: cl(Y) \longrightarrow \R$ such that the pair $(y, p(y))$ is incentive compatible.
	\end{remark}
	
	
	As a corollary of Lemma \ref{incen/convex},  implementable functions can be characterized as $G$-subdifferential of $G$-convex functions. 
	
	
	\begin{corollary}\label{cor:implementable}
		Given Assumption \ref{assmp:Gdecreasing}, a function $y: X \longrightarrow cl(Y)$ is implementable if and only if there exists a $G$-convex function $u(\cdot)$ such that $y(x) \in \partial^G u(x)$ for each $x\in X$.
	\end{corollary}
}

When parameterization of preferences is linear in agent types and price, Corollary \ref{cor:implementable} says that a function is implementable if and only if it is monotone increasing. In general quasilinear cases, this coincides with Proposition 1 of Carlier \cite{Carlier01}. \medskip


\subsection{Reformulation of the Monopolist's Problem}\label{subsection:reformulation}
From the original monopolist's problem \eqref{origin_problem}, we replace product-price pair $(p,y)$ by the value-product pair $(u,y)$, using $u(\cdot) = G(\cdot, y(\cdot), p(y(\cdot)))$. %, since $G$ is strictly monotone in the third variable. 
Combining this with Lemma \ref{incen/convex}, the incentive-compatibility constraint $(y,p(y))$ is equivalent to $G$-convexity of $u(\cdot)$ and $y(x) \in \partial^G u(x)$ for all $x\in X$. Therefore, one can rewrite the monopolist's problem as follows.

\begin{proposition}\label{equiv_form}
	
	Given Assumptions \ref{assmp:Gregular} and \ref{assmp:Gdecreasing}, the monopolist's problem $(P_0)$ is equivalent to
	
	\begin{equation}\label{Principal_new_problem}
	(P)
	\begin{cases}
	\sup \tilde{\Pi}(u,y):=\int_{X} \pi(x, y(x), H(x,y(x), u(x))) d\mu(x)\\
	s.t.\ $u$ \text{ is } $G$\text{-convex };\\
	s.t.\ y(x) \in \partial^G u(x) \text{ and }u(x)\ge u_{\emptyset}(x) \text{ for all } x \in X.
	\end{cases}
	\end{equation}
\end{proposition}

\medskip


\subsection{Other Assumptions}\label{subsection:assumptions}
In the next section, we will show the existence result of the rewritten monopolist's problem $(P)$ given in \eqref{Principal_new_problem}. For preparation of the main result, we introduce the following assumptions. Note that, even in 1-dimensional case, we assume no single-crossing type condition. We also include two propositions here, which will be employed in the proof of Proposition \ref{proposition:convergence}. \medskip%, i.e. continuous differentiable functions.} \medskip


%We adopt technical hypotheses, following from Carlier (Assumptions {assmp:Gcoordinate-monotone}-\ref{assmp:Gtech3}) and Trudinger (Assumptions \ref{assmp:Gregular} - \ref{assmp:Gdecreasing}) as below:\medskip



\begin{comment}
	{ is the so-called twist condition which is the similar/dual to but much less restrictive than the generalized single crossing condition proposed by McAfee-McMillan \cite{McAfeeMcMillan88}. Comparing to Figalli-Kim-McCann \cite{FigalliKimMcCann11}}\medskip
\end{comment}


\begin{assumption}\label{assmp:Gcoordinate-monotone}
	$G$ is coordinate-monotone
	~in $x$. That is, for each $(y,z)\in cl(Y\times Z)$, and for all $ (\alpha, \beta) \in X^2$, if $\alpha_i\ge \beta_i$ for all $ i=1,2,...,M$, then $G(\alpha,y,z)\ge G(\beta, y,z)$.
\end{assumption}


In Assumption \ref{assmp:Gcoordinate-monotone}, we assume that agent utility increases along each consumer attribute coordinate. Given coordinate monotonicity of $G$ in the first variable, one can show that all the $G$-convex functions are nondecreasing. Therefore, the value functions are also monotonic with respect to agents' attributes.\medskip

\begin{proposition}\label{nondecreasing}
	Given Assumption \ref{assmp:Gcoordinate-monotone}, $G$-convex functions are nondecreasing in coordinates.
\end{proposition}



 {In the following, we use  $D_x G(x,y,z) := (\frac{\partial G}{\partial x_1}, \frac{\partial G}{\partial x_2}, \dots, \frac{\partial G}{\partial x_M})(x,y,z)$ to denote derivative with respect to $x$. For any vector in $\R^M$ or $\R^N$, we use $||\cdot||$ and $||\cdot||_{\alpha}$ to denote its Euclidean  $2$-norm and $\alpha$-norm ($\alpha \ge 1$), respectively. For example, for $x\in \R^M$, we have $||x|| = \sqrt{\sum_{i=1}^{M} x_i^2}$ and $||x||_{\alpha} = (\sum_{i=1}^{M} |x_i|^{\alpha})^{\frac{1}{\alpha}}$.}\medskip
 
In Rochet-Chon$\acute{e}$'s model, $H(x,y,u) = x\cdot y -u$ and $\pi(x,y,z) = z-C(y)$, for some superlinear cost function $C$. In this case, $\pi(x,y,H(x,y,u)) = x\cdot y -u -C(y)$. Since $C$ is superlinear and the space $X$ is bounded, it is reasonable to assume the following:\medskip


	\begin{assumption}\label{assmp:Gtech0}
		$\pi(x,y,H(x,y,u))$ is super-linearly decreasing in $y$. That is, there exist $\alpha \ge 1$, $a_1, a_2> 0$ and $b\in \R$, such that $\pi(x,y,H(x,y,u)) \le -a_1 ||y||_{\alpha}^{\alpha} - a_2 u +b$ for all $ (x, y, u)\in \{ (x,y, G(x,y,z))| x\in X, y\in Y, z\in \R \}$; or equivalently, $\pi(x,y,z) + a_2 G(x,y,z) \le -a_1 ||y||_{\alpha}^{\alpha}  +b$ for all $ (x, y, z)\in X\times cl(Y)\times \R$.
	\end{assumption}
	
	
	As shown in the alternative formulation, Assumption \ref{assmp:Gtech0} requires the existence of some weighted surplus which is super-linearly decreasing in product. In the case that $Y$ is bounded, Assumption \ref{assmp:Gtech0} is equivalent to the existence of some weighted surplus bounded from above.   \medskip






Assumptions \ref{assmp:Gtech1} - \ref{assmp:Gtech3} are some technical assumptions on $D_xG$, which are automatically satisfied for $X$, $Y$, $Z$ being compact.\medskip


\begin{assumption}\label{assmp:Gtech1}
	{$D_x G(x,y,z)$ is Lipschitz in $x$},
	uniformly in $(y,z)$, meaning there exists $k$ such that
	$||D_xG(x,y,z)-D_x G(x',y,z)||\le k||x-x'||$ %%FOR SOME $k$ AND 
	%FOR
	for all $(x, x',y, z)\in X^2\times cl(Y) \times cl(Z)$.
\end{assumption}

%---For $G(x,y,z) = \langle x, y \rangle -f(z)$, Assumption \ref{assmp:Gtech1} will be reduced to 

\begin{assumption}\label{assmp:Gtech2}
	 $||D_x G(x,y,z)||_{1}$ increases in $y$. More precisely, there exist $ \beta \in (0, \alpha], c>0$, and $ d\in \R$, such that $||D_x G(x,y,z)||_{1}\le c||y||_{\beta}^{\beta} +d$ for all $ (x, y, z)\in X\times cl(Y) \times cl(Z)$.
\end{assumption}


\begin{assumption}\label{assmp:Gtech3}
	 Coercivity of {$1$-norm} of $(D_xG)$. For all $ s>0$, there exists $r>0$, such that $\sum_{i=1}^{M} |D_{x_i}G(x,y,z)|\ge s$, for all $(x, y, z)\in X\times  cl(Y) \times cl(Z)$, whenever $||y||\ge r$.
\end{assumption}



{Allowing Assumption \ref{assmp:Gcoordinate-monotone}, {the derivatives $D_{x_i}G$ are always nonnegative; therefore,} we no longer need to take absolute values of $D_{x_i}G$ in the inequality of Assumption \ref{assmp:Gtech3}.} And then Assumption~\ref{assmp:Gtech3} says that the marginal utility of agents who select the same product $y$ will increase to infinity as $||y||$ approaches infinity, uniformly for all agents and prices. {For instance, when $M = N$, utility $G(x,y,z) = \sum_{i=1}^{M} x_iy_i^2 -f(z) $ satisfies Assumption~\ref{assmp:Gtech3}, because $\sum_{i=1}^{M} |D_{x_i}G(x,y,z)| = \sum_{i=1}^{M} D_{x_i}G(x,y,z) = \sum_{i=1}^{M} y_i^2 \longrightarrow +\infty$ as $||y|| \longrightarrow + \infty$. In addition, this $G$ also satisfies all the other assumptions.}\medskip


In general, if $Y$ is bounded, any $G$ in the form of $G(x,y,z) = b(x,y) - f(y,z)$, with $b \in C^1(cl(X\times Y))$ and $f\in C^0(cl(Y \times Z))$, satisfies Assumption~\ref{assmp:Gtech1} - \ref{assmp:Gtech3}.\medskip


Proposition \ref{Subdiff/Bdd} presents that uniform boundedness of agents' value functions on some compact subset implies uniform boundedness of corresponding agents' choices of their favorite products. \medskip

\begin{proposition}\label{Subdiff/Bdd}
	Given Assumptions \ref{assmp:Gregular}, \ref{assmp:Gdecreasing}, \ref{assmp:Gcoordinate-monotone}, \ref{assmp:Gtech3}, and let $u(\cdot)$ be a $G$-convex function on $X$, $\omega$ be a compact subset of $X$, $\delta>0, R>0$, satisfying $\omega+\delta\overline{B(0,1)}\subset X$ and $|u(x)|\le R$ for all $x\in \omega + \delta \overline{B(0,1)}$ (here, $\overline{B(0,1)}$ denotes the closed unit ball of $\R^M$). Then, there exists $T = T(\omega,\delta, R) > 0$, such that $||y||\le T$ for any $x \in \omega$ and any $y\in \partial^Gu(x)$.
\end{proposition}




Assumptions~\ref{assmp:Pi1} states constraints on the continuity of principal's profit function $\pi$, integrability of participation constraint $u_{\emptyset}$, and absolutely continuity of measure $\mu$ with respect to the Lebesgue measure. 

%In Assumption , we assume principal's profit would increase as the selling price increases, for any fixed agent and product type. We also assume that profit will be limited for any fixed selling price, uniformly on agent-product type. \medskip

\begin{assumption}\label{assmp:Pi1}
	Profit function $\pi$ is continuous on $cl(X\times Y\times Z)$. The participation constraint $u_{\emptyset}$ is integrable with respect to $d \mu$, where the measure $\mu$ is absolutely continuous with respect to the Lebesgue measure and has $X$ as its support.
\end{assumption}



%%\begin{assumption}\label{assmp:Pi2}
%%	The summation of principal's profits and agents' utilities is uniformly bounded. %Uniform boundedness for the summation of principal's profits and agents' utilities. 
%%	There exists a constant $C_0$, such that $\pi(x, y,z) + G(x,y,z) \le C_0$, for all $(x, y, z) \in X \times  cl(Y) \times cl(Z)$.
%%\end{assumption}
%%
%%In Assumption \ref{assmp:Pi2}, we assume that if an agent $x$ purchases product $y$ at price $z$, then his utility plus principal's benefit will be less than some large number. Otherwise, there will be a sequence of agent-product-price triplets $\{ (x_n, y_n, z_n)\}$, such that the summation approaches infinity as $n$ goes to infinity.\medskip


For $\alpha \ge 1$, denote $L^{\alpha}(X)$ as the space of functions for which the $\alpha$-th power of the absolute value is Lebesgue integrable {with respect to the measure $d \mu$}. That is, a function $f: X\longrightarrow \R$ is in $L^{\alpha}(X)$ if and only if $\int_{X} |f|^{\alpha} d\mu <+\infty$. For instance, Assumption \ref{assmp:Pi1} implies $u_{\emptyset}\in L^{1}(X)$.\medskip


%{\bf Assumption 10.} Profit function $\pi(x,y,z)$ is strictly increasing with respect to the third variable $z$, for each $(x,y)\in X\times Y$.\medskip

 

%{\bf Assumption 11.} For each $z < +\infty $, there exists a constant $C_0$, such that $\pi(x, y,z) \le C_0$, for all $(x, y) \in X \times Y$.\medskip
 







\bigskip





\bigskip







\section{Main result}\label{section:mainresult}
In this section, we state the existence theorem, the proof of which is provided in the Section \ref{section:proofs}. 


\begin{theorem}[Existence]
	Under Assumptions \ref{assmp:Gregular} - \ref{assmp:Pi1},  assume $\mu$ is equivalent to the Lebesgue measure on $X$, then the monopolist's problem $(P)$ admits at least one solution.
\end{theorem}

Technically, in order to demonstrate existence, we start from a sequence of value-product pairs, whose total profits have a limit that is equal to the supreme of $(P)$.
Then we need to show that this sequence converges, up to a subsequence, to a pair of limit mappings. Then we show this limit value-product pair satisfies the constraints of $(P)$; and its corresponding total payoff is better (or no worse) than those of any other admissible pairs. \medskip

In the following, we denote $W^{1,1}(X)$ as the Sobolev space of $L^1$ functions whose first derivatives exist in the weak sense and belong to $L^1(X)$. For more properties of Sobolev spaces and weak derivatives, see Evans~\cite[Chapter 5]{Evans98}. If $\omega$ is some open subset of $X$, notation $\omega \subset \subset X$ means the closure of $\omega$ is also included in $X$.\medskip

Lemma \ref{lemma1} provides sequence convergence results of convex functions, which are uniformly bounded in Sobolev spaces on open convex subsets. We state this classical result without proof, which can be found in Carlier \cite{Carlier01}.\medskip


\begin{lemma}\label{lemma1}
	Let $\{u_n\}$ be a sequence of convex functions in $X$ such that, for every open convex set $\omega \subset \subset X$, the following holds:
	\begin{equation*}
	\sup\limits_{n} ||u_n||_{W^{1,1}(\omega)} < +\infty
	\end{equation*}
	Then, there exists a function $u^*$ %.. $\bar{u}$ means indifference curve / specific utility level} 
	which is convex in $X$, a measurable subset $A$ of $X$ and a subsequence again labeled $\{u_n\}$ such that\\
	1. $\{u_n\}$ converges to $u^*$ uniformly on compact subsets of $X$;\\
	2. $\{\nabla u_n\}$ converges to $\nabla u^*$ pointwise in $A$ and $dim_{H}(X\setminus A)\le M-1$, where $dim_{H}(X\setminus A)$ is the Hausdorff dimension of $X\setminus A$.
\end{lemma}

{We extend the above convergence result to $G$-convex functions in the following proposition, which is required in the proof of the Existence Theorem, as it extracts a limit function from a converging sequence of value functions.}

\begin{proposition}\label{proposition:convergence}
	Assume Assumptions \ref{assmp:Gregular}, \ref{assmp:Gdecreasing}, \ref{assmp:Gcoordinate-monotone}, \ref{assmp:Gtech1}, \ref{assmp:Gtech3}, and let $\{u_n\}$ be a sequence of $G$-convex functions in $X$ such that for every open convex set $\omega \subset \subset X$, the following holds:
	\begin{equation*}
	\sup\limits_{n} ||u_n||_{W^{1,1}(\omega )} < +\infty
	\end{equation*}
	Then there exists a function $u^*$ which is  $G$-convex in $X$, a measurable subset $A$ of $X$, and a subsequence again labeled $\{u_n\}$ such that\\
	1. $\{u_n\}$ converges to $u^*$ uniformly on compact subsets of $X$;\\
	2. $\{\nabla u_n\}$ converges to $\nabla u^*$ pointwise in $A$ and $dim_{H}(X\setminus A)\le M-1$.
\end{proposition}

In the proof of Proposition \ref{proposition:convergence}, we show that the sequence of $G$-convex functions is convergent by applying results from Lemma \ref{lemma1}, then prove that the limit function is also $G$-convex. \medskip



\bigskip





\section{Future Work}\label{section:futurework}

We have strong interest in giving an explicit solution for the non-quasilinear example on the real line and in high dimension. We also would like to investigate, among other things, the conditions under which the matching map $y: X \longrightarrow cl(Y)$ is continuous and/or differentiable. Given the technical arguments employed in this paper, it may be very fruitful to study possible generalizations of other known results for convex functions to $G$-convex functions.




\bigskip



\section{Proofs}\label{section:proofs}

\begin{proof}[Proof of Proposition \ref{lemma_continuity}]
	(Proof by contradiction). Suppose $H$ is not continuous, then there exists a sequence ${(x_n, y_n, z_n)} \subset cl(X\times Y \times Z)$ converging to $(x, y, z)$ and $\varepsilon >0$ such that $|H(x_n, y_n, z_n) - H(x,y,z)|>\varepsilon$ for all $n\in \N$. Without loss of generality, we assume $H(x_n, y_n, z_n) - H(x,y,z)>\varepsilon$ for all $n\in \N$. Therefore, we have $H(x_n, y_n, z_n) > H(x,y,z)+\varepsilon$. By Assumption \ref{assmp:Gdecreasing}, this implies $z_n < G(x_n, y_n, H(x,y,z)+\varepsilon)$ for all $n\in \N$. Taking limit $n\longrightarrow \infty$ at both sides, since $G$ is continuous from Assumption \ref{assmp:Gregular}, we have $z \le G(x, y, H(x,y,z)+\varepsilon)$. It implies $H(x,y,z) \ge H(x,y,z)+\varepsilon$. Contradiction!
\end{proof}

\vspace{0.3cm}

\begin{proof}[Proof of Lemma \ref{convex-subdiff}]
$"\longrightarrow"$. Assume $u$ is $G$-convex, we want to show $u$ is $G$-subdifferentiable everywhere, i.e., the statement $\partial^G u(x_0)\neq \emptyset$ is true for any $ x_0\in X$.

Since $u$ is $G$-convex, by definition, for any fixed $x_0 \in X$, there exist $y_0$ and $z_0$, such that $u(x_0) = G(x_0,y_0,z_0)$, and for any $x \in X$, $u(x)\ge G(x, y_0, z_0)$. By definition of function $H$, we know $z_0 = H(x_0,y_0,u(x_0))$. Therefore, combining the above inequality, we have $u(x) \ge G(x, y_0, H(x_0,y_0,u(x_0)))$, for any $x \in X$. By definition of $G$-subdifferentiability, it implies $y_0 \in \partial^G u(x_0)$, that is, $\partial^G u(x_0) \neq \emptyset$.


$"\Leftarrow"$. Assume $u$ is $G$-subdifferentiable everywhere. Then for any fixed $x_0 \in X$, there exists $y_0 \in \partial^G u(x_0)$. Define $z_0:=H(x_0,y_0,u(x_0))$, then by definition of function $H$, one has $u(x_0) = G(x_0, y_0, z_0)$.
%\sloppy

Since $y_0\in \partial^G u(x_0)$, by definitions of $G$-subdifferentiability, we have $u(x)\ge G(x,y_0,H(x_0,y_0,u(x_0)))$\ $ = G(x,y_0,z_0)$, for any $x\in X$, where the last equality comes from the definition of $z_0$.

By definition (of $G$-convexity), $u$ is $G$-convex.
\end{proof}
\vspace{0.3cm}
\begin{proof}[Proof of Lemma \ref{incen/convex}]
$"\longrightarrow"$. Suppose $(y,z)$ is incentive compatible. For any fixed $x_0 \in X$, let $y_0 = y(x_0)$ and $z_0 = z(x_0)$. Then $u(x_0) = G(x_0, y(x_0), z(x_0)) = G(x_0, y_0, z_0)$. By incentive compatibility of the contract $(y,z)$, for any $x\in X$, one has $G(x, y(x), z(x)) \ge G(x, y(x_0), z(x_0))$. This implies $u(x) \ge G(x,y_0,z_0)$, for any $x\in X$, since $u(x)= G(x, y(x), z(x))$,  $y_0 = y(x_0)$ and $z_0 = z(x_0)$. By definition, $u$ is $G$-convex. 

Since $u(x_0)=G(x_0, y_0, z_0)$, by definition of function $H$, one has $z_0 = H(x_0, y_0, u(x_0))$.  Combining with $u(x) \ge G(x, y_0, z_0)$,  for any $x\in X$, which is concluded from above, we have $u(x)\ge G(x, y_0, H(x_0, y_0, u(x_0)))$, for any $x\in X$. By definition of  $G$-subdifferentiability, one has $y_0 \in \partial^G u(x_0)$, and thus $y(x_0) = y_0 \in \partial^G u(x_0)$.

$"\Leftarrow"$. Assume that $u = G(x, y(x),z(x))$ is $G$-convex, and $y(x)\in \partial^G u(x)$, for any $x\in X$. For any fixed $x \in X$, since $y(x)\in \partial^G u(x)$, for any $x'\in X$, one has 
\begin{equation}\label{eqn_prop3.4}
	u(x')\ge G(x', y(x), H(x, y(x), u(x)))
\end{equation} 
Since $u(x) = G(x, y(x),z(x))$, by definition of function $H$, one has $z(x) = H(x,y(x), u(x))$. Combining with the inequality \eqref{eqn_prop3.4}, we have $u(x')\ge G(x', y(x), z(x))$. Noticing $u(x') = G(x',y(x'),$ $z(x')) $, one has $G(x',y(x'),z(x')) = u(x') \ge G(x', y(x), z(x))$.
By definition, (y,z) is incentive compatible.
\end{proof}

\vspace{0.3cm}

\begin{proof}[Proof of Remark \ref{rmk:implementability}]
	One direction is easier: given $p$ and $y$, define $z(\cdot):= p(y(\cdot))$. Then the conclusion follows directly. \medskip
	
	Given an incentive-compatible pair $(y, z): X \longrightarrow cl(Y) \times \R$, we need to construct a price menu $p: cl(Y)\longrightarrow \R$. If $y= y(x)$ for some $x\in X$, define $p(y):= z(x)$; for any other $y \in cl(Y)$, define $p(y) := \bar{z}$. \medskip
	
	We first show $p$ is well-defined. Suppose $y(x) = y(x')$ with $x\ne x'$, from incentive compatibility of $(p,y)$, we have $G(x,y(x), z(x)) \ge G(x, y(x'), z(x')) = G(x, y(x), z(x'))$. Since $G$ is strictly decreasing on its third variable, the above inequality implies $z(x) \le z(x')$. Similarly, one has $z(x) \ge z(x')$. Therefore, $z(x) = z(x')$ and thus $p$ is well-defined. \medskip
	
	The incentive compatibility of $(y, p(y))$ follows from that of $(y, z)$ and definition of $p$.
\end{proof}
\vspace{0.3cm}
\begin{proof}[Proof of Corollary \ref{cor:implementable}]
	One direction is immediately derived from the definition of implementability and Lemma \ref{incen/convex}.\medskip
	
	Suppose there exists some convex function $u$ such that $y(x) \in \partial^G u(x)$ for each X. Define $z(\cdot):= H(\cdot, y(\cdot), u(\cdot))$, then $u(x) = G(x, y(x), z(x))$.
	Lemma \ref{incen/convex} implies $(y, z)$ is incentive compatible, and thus $y$ is implementable.
\end{proof}
\vspace{0.3cm}

\begin{proof} [Proof of Proposition \ref{equiv_form}] We need to prove both directions for equivalence of $(P_0)$ and $(P)$.\medskip
	
	1. For any incentive-compatible pair $(y, p(y))$, define $u(\cdot) := G(\cdot,y(\cdot), p(y(\cdot)))$. Then by Lemma \ref{incen/convex}, we have $u(\cdot)$ is $G$-convex and $y(x) \in \partial^G u(x)$ for all $x \in X$. From the participation constraint, $G(x, y(x), p(y(x))) \ge u_{\emptyset}(x)$ for all $x\in X$. This implies $u(x)\ge u_{\emptyset}(x)$ for all $x\in X$. Besides, two integrands are equal: $\pi(x, y(x), p(y(x))) = \pi(x,y(x), H(x,y(x), u(x)))$. Therefore, $(P_0) \le (P)$.\medskip
	
	2. On the other hand, assume $u(\cdot)$ is $G$-convex, $y(x)\in \partial^G u(x)$ and $u(x) \ge u_{\emptyset}(x)$ for all $x \in X$. From Corollary \ref{cor:implementable} and Remark \ref{rmk:implementability}, we know $y$ is implementable and there exists a price menu $p: cl(Y) \longrightarrow \R$, such that the pair $(y, p(y))$ is incentive compatible, where $p(y) = H(x,y(x), u(x))$ for $y = y(x) \in y(X) :=\{ y(x) \in cl(Y) | x \in X \}$; $p(y) = \bar{z}$ for other $y\in cl(Y)$. Firstly, the mapping $p$ is well-defined, using the same argument as that in Remark \ref{rmk:implementability}. %Then by Lemma \ref{incen/convex}, $(y, p(y))$ is incentive compatible. 
	Secondly, the participation constraint holds since $G(x,y(x), p(y(x))) = u(x) \ge u_{\emptyset}(x)$ for all $x\in X$. \medskip
	
	Thirdly, let us show this price menu $p$ is lower semicontinuous. Let $\tilde{p}$ be the restriction of $p$ on $y(X)$. Suppose that $\{y_k \} \subset y(X)$ converges $y_0 \in y(X)$ with $y_k = y(x_k)$ and $y_0 = y(x_0)$, satisfying $\lim\limits_{k \longrightarrow \infty} \tilde{p}(y_k) = \liminf\limits_{y \longrightarrow y_{0}} \tilde{p}(y)$.  Let $z_{k}:= \tilde{p}(y_k)$ and $z_{\infty}:=\lim\limits_{k \longrightarrow \infty} z_k$. To prove lower semicontinuity of $\tilde{p}$, we need to show $\tilde{p}(y_0)\le z_{\infty}$.  Since $y_k \in \partial^G u(x_k)$, we have $u(x) \ge G(x,y_k, H(x_k, y_k, u(x_k))) = G(x, y_k, z_k)$. Taking $k\longrightarrow \infty$, we have $u(x)\ge G(x, y_0, z_{\infty})$. This implies $G(x_0, y_0, \tilde{p}(y_0)) = u(x_0) \ge G(x_0, y_0, z_{\infty})$. By Assumption \ref{assmp:Gdecreasing}, we know $\tilde{p}(y_0) \le z_{\infty}$. Thus $\tilde{p}$ is lower semicontinuous. Since $p$ is an extension of $\tilde{p}$ from $y(X)$ to $cl(Y)$ as its lower semicontinuous hull, satisfying $v(y)= \bar{z}$ for all $y\in cl(Y)\setminus y(X)$, we know $p$ is also lower semicontinuous.
	\medskip
	
	Lastly, two integrands are equal: $\pi(x, y(x), p(y(x))) = \pi(x,y(x), H(x,y(x), u(x)))$. Therefore, $(P_0) \ge (P)$.
\end{proof}


\vspace{0.3cm}


\begin{proof}[Proof of Proposition \ref{nondecreasing}]
	Let $u$ be any $G$-convex function, and let $\alpha$, $\beta$ be any two agent types in $X$ with $\alpha \ge \beta$. By $G$-convexity of $u$, for this $\beta$, there exist $y\in cl(Y)$ and $z \in cl(Z)$, such that $u(\beta)=G(\beta, y,z)$ and $u(x)\ge G(x, y,z)$, for any $x\in X$. Since $\alpha \ge \beta$, by Assumption \ref{assmp:Gcoordinate-monotone}, we have $G(\alpha, y,z)\ge G(\beta,y,z)$. Combining with $u(\alpha)\ge G(\alpha, y,z)$ and $u(\beta) = G(\beta,y,z)$, one has $u(\alpha) \ge u(\beta)$. Thus, $u$ is nondecreasing.
\end{proof}

\vspace{0.3cm}


\begin{proof}[Proof of Proposition \ref{Subdiff/Bdd}]
	(Proof by contradiction).\medskip
	By Assumption \ref{assmp:Gcoordinate-monotone} and Assumption \ref{assmp:Gtech3}, for $s=\frac{4R\sqrt{M}}{\delta}$, there exists $r>0$, such that for any $(x, y, z)\in X \times  cl(Y) \times cl(Z)$, whenever $||y||\ge r$, we have $\sum\limits_{i=1}^{M}D_{x_i}G(x,y,z)\ge \frac{4R\sqrt{M}}{\delta}$.\medskip
	
	Assume the boundedness conclusion of this proposition is not true. Then for this $r$, there exist $ x_0 \in \omega$ and  $ y_0\in \partial^G u(x_0)$, such that $||y_0||\ge r$. Thus,
	\begin{equation}\label{eqn_coercivity}
		\sum\limits_{i=1}^{M}D_{x_i}G(x,y_0,z)\ge \frac{4R\sqrt{M}}{\delta}, \ \ \ \ \ \text{ for all } x \in X, z \in \R.
	\end{equation}
	Since $y_0 \in \partial^G u(x_0)$, by definition of $G$-subdifferential, we have $u(x)\ge G(x,y_0,H(x_0,y_0,u(x_0)))$,  for any $ x \in X$. Take $x=x_0+\delta x_{-1}$, where $x_{-1}:=(\frac{1}{\sqrt{M}}, \frac{1}{\sqrt{M}}, \cdots, \frac{1}{\sqrt{M}})$ is a unit vector in $\R^M$ with each coordinate equal to $\frac{1}{\sqrt{M}}$. Then 
	\begin{equation}\label{eqn_prop3.6}
		u(x_0+\delta x_{-1})\ge G(x_0+\delta x_{-1},y_0,H(x_0,y_0,u(x_0))).
	\end{equation}
	For any $x \in \omega+ \delta \overline{B(0,1)}$, from conditions in the proposition, we have $||u(x)||\le R$. Therefore, 
	\begin{flalign*}
	2R &\ge |u(x_0+\delta x_{-1})|+|u(x_0)|&&\\
	&\ge |u(x_0+\delta x_{-1})- u(x_0)|&& \text{(By Triangular Inequality)}\\
	&\ge u(x_0+\delta x_{-1}) - u(x_0)&& \\
	&\ge G(x_0+\delta x_{-1}, y_0, H(x_0,y_0,u(x_0)))&& \text{(By Inequality \eqref{eqn_prop3.6})  and (By definition }\\
	& -G(x_0, y_0, H(x_0, y_0, u(x_0))) && \text{of $H$, $u(x_0) = G(x_0, y_0, H(x_0, y_0, u(x_0)))$}\\
	&= \int_{0}^{1}\delta \langle x_{-1},  D_{x}G(x_0+t\delta x_{-1}, y_0, H(x_0,y_0,u(x_0)))\rangle dt&& \text{(By Fundamental Theorem of Calculus)}\\
	&= \frac{\delta}{\sqrt{M}}\int_{0}^{1} \sum\limits_{i=1}^{M}D_{x_i}G(x_0+t\delta x_{-1}, y_0, H(x_0, y_0, u(x_0))) dt&&\\
	&\ge \frac{\delta}{\sqrt{M}}\int_{0}^{1}\frac{4R\sqrt{M}}{\delta}dt&& \text{(By Inequality \eqref{eqn_coercivity})}\\
	&= \frac{\delta}{\sqrt{M}}\cdot\frac{4R\sqrt{M}}{\delta}&&\\
	&= 4R&&
	\end{flalign*}
	Contradiction!
	Thus, our assumption is wrong. The boundedness conclusion of this proposition is true. That is, there exists $T>0$, such that for any $x \in \omega$, $y \in \partial^G u(x)$, one has $||y||\le T$. In addition, here $T = T(\omega, \delta, R)$ is independent of $u$. In fact, from the above argument, we can see that $T \le r$, which does not depend on $u$.
\end{proof}

\vspace{0.3cm}



\begin{proof}[Proof of Proposition \ref{proposition:convergence}]
	Assume $\{u_n\}$ is a sequence of $G$-convex functions in $X$ such that for every open convex set $\omega \subset \subset X$, the following holds:
	\begin{equation*}
	\sup\limits_{n} ||u_n||_{W^{1,1}(\omega )} < +\infty.
	\end{equation*}

	{\bf Step 1:} By Assumption \ref{assmp:Gtech1}, there exists $k>0$, such that for any $(x, x')\in X^2$, $y\in cl(Y)$ and $z\in cl(Z)$, one has $||D_xG(x,y,z)-D_x G(x',y,z)||\le k||x-x'||$. Denote $G_{\lambda}(x,y,z) := G(x,y,z)+\lambda||x||^2$, where $\lambda \ge \frac{1}{2}\Lip(D_xG)$, with $\Lip(D_xG)
	:=\sup\limits_{\{(x,x',y,z)\in X\times X\times  cl(Y) \times cl(Z):~x \neq x'\}} \frac{||D_xG(x,y,z)-D_x G(x',y,z)||}{||x-x'||}$. %\le k$.
	
	Then, for any $(x, x')\in X^2$, by Cauchy–Schwarz inequality, one has 
	\begin{flalign*}
		& \langle D_xG_{\lambda}(x,y,z)-D_x G_{\lambda}(x',y,z) , x-x'\rangle &&\\
		= & ~\langle D_xG(x,y,z)-D_x G(x',y,z) , x-x'\rangle + 2\lambda ||x-x'||^2 && \text{(By Defition of $G_{\lambda}(x,y,z)$)}\\
		\ge & ~-||D_xG(x,y,z)-D_x G(x',y,z)|| ||x-x'||+ 2\lambda ||x-x'||^2 &&\text{(By Cauchy–Schwarz Inequality)}\\
		\ge & ~[2\lambda - \Lip(D_xG)]||x-x'||^2 &&\text{(By Definition of $\Lip(D_xG)$)}\\
		\ge & ~0.&&
	\end{flalign*} 
	
	Thus, $G_{\lambda}(\cdot, y, z)$ is a convex function on $X$, for any fixed $(y, z) \in cl(Y) \times cl(Z).$\medskip
	
{\bf Step 2:}	Since $u_n$ is $G$-convex, by Lemma \ref{convex-subdiff}, we know $$u_n(x) = \max\limits_{x'\in X, y\in \partial^G u_n(x')} G(x,y,H(x',y,u_n(x'))).$$ 
Define $v_n(x):= u_n(x) +\lambda ||x||^2$. Then 
\begin{flalign*}
	v_n(x) =& \max\limits_{x'\in X, y\in \partial^G u_n(x')}G(x,y,H(x',y,u_n(x'))) +\lambda ||x||^2 \\
	=& \max\limits_{x'\in X, y\in \partial^G u_n(x')}(G(x,y,H(x',y,u_n(x'))) +\lambda ||x||^2)\\
	=& \max\limits_{x'\in X, y\in \partial^G u_n(x')} G_{\lambda}(x,y,H(x',y,u_n(x'))).	
\end{flalign*}

	
	Since $G_{\lambda}(\cdot,y,H(x',y,u_n(x')))$ is convex for each $(x', y)$, we have $v_n(x)$, as supremum of convex functions, is also convex, for all $n \in \N$.\medskip
	
	
	
{\bf Step 3:}	Since $v_n:= u_n +\lambda||x||^2$ and $\sup\limits_{n}||u_n||_{W^{1,1}(\omega)} < +\infty$, one has $\sup\limits_{n}||v_n||_{W^{1,1}(\omega)} < +\infty$, for any $\omega \subset \subset X$. Hence $\{v_n\}$ satisfies all the assumptions of Lemma \ref{lemma1}. So, by conclusion of Lemma \ref{lemma1}, there exists a convex function $v^*$ in $X$ and a measurable set $A \subset X$, such that $dim (X \setminus A)\le M-1$ and up to a subsequence, $\{v_n\}$ converges to $v^*$ uniformly on compact subset of $X$ and $(\nabla v_n)$ converges to $\nabla v^*$ pointwise in A.
	
	Let $u^*(x):=v^*(x)-\lambda||x||^2$, then  $(u_n)$ converges to $u^*$ uniformly on compact subset of $X$ and $(\nabla u_n)$ converges to $\nabla u^*$ pointwise in A.\medskip
	
{\bf Step 4:}	Finally, let us prove that $u^*$ is $G$-convex.\medskip
	Define $\Gamma(x):=\cap_{i\ge 1}\overline{\cup_{n\ge i}\partial^G u_n(x)}$, for all $x\in X$.\medskip
	%$\tilde{u}(x):=\sup\limits_{x'\in X, y\in \Gamma (x')}G(x,y,H(x',y,u^*(x')))$, where\medskip 
	
	
{\bf	Step 4.1.} Claim: For any $x'\in X$, we have $\Gamma(x') \neq \emptyset$.\medskip
	
	Proof of this Claim: 
	
	{\bf Step 4.1.1.} Let us first show for any $\omega \subset\subset X$, $\sup\limits_{n}||u_n||_{L^{\infty}(\bar{\omega})}<+\infty$.
	
	If not, then there exits a sequence $\{x_n\}_{n=1}^{\infty}\subset \bar{\omega}$, such that $\limsup\limits_{n}|u_n(x_n)|=+\infty$.
	
	Since $\bar{\omega}$ is compact, there exists $\bar{x}\in \bar{\omega}$, such that, up to a subsequence, $x_n\longrightarrow \bar{x}$. Again up to a subsequence, we may assume that $u_n(x_n)\longrightarrow +\infty$.
	
	Since $\bar{x} \in \bar{\omega} \subset \subset X$, there exists $\delta >0$, such that $\bar{x}+\delta x_{-1} \in X$, where $x_{-1}:=(\frac{1}{\sqrt{M}}, \frac{1}{\sqrt{M}}, \cdots, \frac{1}{\sqrt{M}})$ is a unit vector in $\R^M$ with each coordinate equal to $\frac{1}{\sqrt{M}}$. For any $x>\bar{x} + \delta x_{-1}$, there exists $n_0$, such that for any $n>n_0$, we have $x>x_n$. By Proposition \ref{nondecreasing}, $u_n$ are nondecreasing, and thus
	\begin{equation}\label{eqn_integral}
	\int_{\{x\in X, x>\bar{x} +\delta x_{-1}\}} u_n(x)dx \ge m\{x\in X, x> \bar{x}+\delta x_{-1}\} u_n(x_n)\longrightarrow +\infty
	\end{equation}
	where $m\{x\in X, x>\bar{x}+\delta x_{-1}\}$ denotes Lebesgue measure of $\{x\in X, x>\bar{x}+\delta x_{-1}\}$, which is positive.
	
	Therefore, we have $||u_n||_{W^{1,1}(\omega')} \ge ||u_n||_{L^{1}(\omega')} \ge \int_{\omega'} u_n(x) dx \longrightarrow +\infty$. This implies $\sup\limits_{n} ||u_n||_{W^{1,1}(\omega')} = +\infty$.
	
	On the other hand, denote $\omega' := \{x\in X|~ x>\bar{x}+\delta x_{-1}\}$, then $\omega' = X \cap \{x\in \R^M|~ x> \bar{x}+\delta x_{-1} \}$. Since both $X$ and $\{x\in \R^M|~ x> \bar{x}+\delta x_{-1} \}$ are open and convex, we have $\omega'$ is also open and convex. Therefore, by assumption, we have $\sup\limits_{n} ||u_n||_{W^{1,1}(\omega')} < +\infty.$ 
	
	Contradiction! Thus, for any $\omega \subset \subset X$, we have $\sup\limits_{n}||u_n||_{L^{\infty}(\bar{\omega})}<+\infty$.\medskip
	
	{\bf Step 4.1.2.} For any fixed $x'\in X$, there exists an open set $\omega \subset \subset X$ and $\delta>0$, such that $x'\in \omega$ and $\omega + \delta \overline{B(0,1)} \subset \subset X$.
	
	From Step 4.1.1, we know  $\sup\limits_{n}||u_n||_{L^{\infty}(\omega + \delta \overline{B(0,1)})} < +\infty$. 
	So there exists $R>0$, such that for all $n\in \N$, we have $|u_n(x)|\le R$, for all $x \in \omega + \delta \overline{B(0,1)}$. Since $u_n$ are $G$-convex functions, by Proposition~\ref{Subdiff/Bdd}, there exists $T = T(\omega, \delta, R) >0$, independent of $n$, such that $||y||\le T$, for any $y \in \partial^G u_n(x')$ and any $n\in \N$. Thus, there exists a sequence $\{ y_n \}$, such that $y_n \in \partial^G u_n(x')$ and $||y_n||\le T$, for all $n\in \N$.
	
	By compactness theorem for sequence $\{ y_n \}$,  there exists $y'$, such that, up to a subsequence, $y_n \longrightarrow y'$. Thus, we have $y' \in \overline{\cup_{n\ge i}\partial^G u_n(x')}$, for all $i\in \N$. It implies $y' \in \cap_{i\ge 1} \overline{\cup_{n\ge i}\partial^G u_n(x')} = \Gamma (x')$. 
	
	Therefore $\Gamma(x') \neq \emptyset$, for all $x' \in X$.\medskip
	
%%%	%By definition of $\tilde{u}$, $\tilde{u}(x)\ge G(x,y, H(x,y,u^*(x))) =  u^*(x)$. {\color{} Since both $X$ and $\Gamma(x')$ are compact}, for any $x_0 \in X$, there exist $x_0'\in X$, $y_0\in \Gamma(x_0')$, such that $\tilde{u}(x_0) = G(x_0, y_0, H(x_0', y_0, u^*(x_0')))$. And for any $x \in X$, $\tilde{u}(x) = \sup\limits_{x'\in X, y\in \Gamma (x')} G(x, y, H(x',y,u^*(x') )) \ge G(x, y_0, H(x_0', y_0, u^*(x_0')))$. By definition, $\tilde{u}$ is $G$-convex. \medskip
	
	
	{\bf Step 4.2.} Now for any fixed $x\in X$, and any $y\in \Gamma(x)$, by Cantor's diagonal argument, there exists $\{y_{n_k}\}_{k=1}^{\infty}$, such that $y_{n_k} \in \partial^G u_{n_k}(x)$ and $\lim\limits_{k\longrightarrow \infty} y_{n_k} = y$.
	For any $k\in \N$, by definition of $G$-subdifferentiability,
	$u_{n_k}(x')\ge G(x', y_{n_k}, H(x, y_{n_k}, u_{n_k}(x)))$, for any $x' \in X$. Take limit $k \longrightarrow \infty$ at both sides, we get $u^*(x') \ge G(x', y, H(x, y, u^*(x)))$, for any $x'\in X$. Here we use the fact that both functions $G$ and
		$H$ are continuous by Assumption \ref{assmp:Gregular} and Proposition \ref{lemma_continuity}.
		%since $G$ is continuous and strictly decreasing with respect to its third variable.} 
Then by definition of $G$-subdifferentiability, the above inequality implies $y\in \partial ^G u^*(x)$. 

	So $\partial^G u^*(x)\neq \emptyset$, for any $x\in X$, which means $u^*$ is G-subdifferentiable everywhere. By Lemma \ref{convex-subdiff}, $u^*$ is $G$-convex.
\end{proof}

\vspace{0.3cm}

\begin{proof}[Proof of the Existence Theorem]
{\bf Step 1:} 	 Define $\Phi_u: x \longmapsto argmin_{\partial^G u(x)} -\pi(x, \cdot, H(x,\cdot,u(x)))$, then by Proposition \ref{Subdiff/Bdd}, the measurable section theorem (cf. \cite[Theorem 1.2, Chapter VIII]{EkelandTemam76}) and Lusin Theorem, one has $\Phi_u$ admits measurable selections. \medskip

Let $\{(u_n, y_n)\}$ be a maximizing sequence of $(P)$, where maps $u_n: X\longrightarrow \R$ and $y_n: X\longrightarrow cl(Y)$, for all $n\in \N$. Without loss of generality, we may assume that for all $n$, $y_n(\cdot)$ is measurable and $y_n(x) \in \Phi_{u_n}(x)$, for each $x\in X$. Starting from $\{(u_n, y_n)\}$, we would find an value-product pair $(u^*, y^*)$ satisfying all the constraints in \eqref{Principal_new_problem}, and show that it is actually a maximizer.\medskip
	
{\bf Step 2:} From Assumption \ref{assmp:Gtech0}, there exist $\alpha \ge 1$, $a_1, a_2> 0$ and $b\in \R$,  such that for each $x\in X$ and $n \in \N$,
\begin{flalign*}
	a_1 ||y_n(x)||_{\alpha}^{\alpha} \le & -\pi(x,y_n(x),H(x, y_n(x), u_n(x))) - a_2 u_n(x) +b \\
	\le &  -\pi(x,y_n(x),H(x, y_n(x), u_n(x)))- a_2 u_{\emptyset}(x) + b,
\end{flalign*}
%\begin{flalign*}
%	a ||y_n(x)||_{\alpha}^{\alpha}	 \le & G(x,y_n(x),H(x,y_n(x),u_n(x))) -b \\
%	\le & C_0 - \pi(x,y_n(x),H(x,y_n(x),u_n(x))) -b, 
%\end{flalign*}
where the second inequality comes from $u_n\ge u_{\emptyset}$. Together with Assumption \ref{assmp:Pi1}, this implies $\{y_n\}$ is bounded in $L^{\alpha}(X)$.\medskip


	By participation constraint and Assumption \ref{assmp:Gtech0}, we know 
	\begin{equation*}
	u_{\emptyset}(x) \le u_n(x) = G(x,y_n(x),H(x,y_n(x),u_n(x))) \le \frac{1}{a_2}(b - \pi(x,y_n(x),H(x,y_n(x),u_n(x)))).
	\end{equation*}



Together with Assumption \ref{assmp:Pi1}, we know $\{u_n\}$ is bounded in $L^1(X)$.\medskip

	By $G$-subdifferentiability, $Du_n(x) = D_x G(x, y_n(x), H(x,y_n(x),u_n(x)))$. By Assumption \ref{assmp:Gtech2}, we have $||Du_n||_{1}\le c||y_n||_{\beta}^{\beta}+d \le c(N+||y_n||_{\alpha}^{\alpha})+d$. The last inequality holds because $\beta \in (0, \alpha]$. Because $X$ is bounded and $\{y_n\}$ is bounded in $L^{\alpha}(X)$, we know $\{Du_n\}$ is bounded in $L^1(X)$.\medskip
	
	Since both $\{u_n\}$ and $\{Du_n\}$ are bounded in $L^1(X)$, one has $\{u_n\}$ is bounded in $W^{1,1}(X)$. By Proposition \ref{proposition:convergence}, there exists a $G$-convex function $u^*$ on $X$, such that, up to a subsequence, $\{u_n\}$ converges to $u^*$ in $L^1$ and uniformly on compact subset of $X$, and $\nabla u_n$ converges to $\nabla u^*$ almost everywhere.\medskip

{\bf Step 3: } Denote $y^*(x)$ as a measurable selection of $\Phi_{u^*}$. Let us show $(u^*,y^*)$ is a maximizer of the principal's program $(P)$. \medskip

	{\bf Step 3.1: }By Assumption \ref{assmp:Gtech0}, for all $x$, $y_n(x)$ and $u_n(x)$,
	%there exists some constant $C_0$, such that for all $x\in X$ and $n\in \N$, we have $\pi(x,y_n(x),H(x,y_n(x), u_n(x)))+G(x,y_n(x),H(x,y_n(x), u_n(x)))\le C_0$. Therefore, 
	one has
	\begin{flalign*}
	&-\pi(x,y_n(x),H(x,y_n(x), u_n(x)))\\
	\ge & \ a_2 G(x,y_n(x),H(x,y_n(x), u_n(x))) -b \\
	=&\  a_2 u_n(x) - b \\
	\ge&\ a_2 u_{\emptyset}(x) - b.
	\end{flalign*}


By Assumption \ref{assmp:Pi1}, $u_{\emptyset}$ is measurable, thus one can apply Fatou's Lemma and get
%we have $\pi(x, y_n(x), H(x, y_n(x), u_n(x))) \le  C_0$, for all .

	\begin{align}\label{3}
	\begin{split}
	\sup \tilde{\Pi}(u,y) & = \limsup\limits_{n} \tilde{\Pi}(u_n, y_n) \\
	&= -\liminf\limits_{n} \int_{X} - \pi(x, y_n(x), H(x,y_n(x),u_n(x)))  ~d\mu(x)\\
	& \le - \int_{X} \liminf\limits_{n} - \pi(x, y_n(x), H(x,y_n(x),u_n(x)))~ d\mu(x). \\
	\end{split}
	\end{align}
	
Let $\gamma(x):=\liminf\limits_{n} - \pi(x, y_n(x), H(x,y_n(x),u_n(x)))$. 
For each $x\in X$, by extracting a subsequence of $\{y_{n} \}$, which is denoted as $\{y_{n_x}\}$, we assume $\gamma(x) = \lim\limits_{n_x} - \pi(x, y_{n_x}(x), H(x,y_{n_x}(x),u_{n_x}(x)))$. \medskip
	
	{\bf Step 3.2: } 	For any fixed $x \in X$, since $u_{n_x}$ are $G$-convex functions and $\{u_{n_x}\}$ is bounded in $L^1(X)$% and bounded from below by a $L^1$ function
	, by Proposition \ref{nondecreasing}, it is also bounded in $L_{loc}^{\infty}(X)$. 	Then by Proposition \ref{Subdiff/Bdd}, $\{y_{n_x}\}$ is also bounded in $L_{loc}^{\infty}(X)$ . Thus there exists a subsequence of $\{y_{n_x}(x)\}$, again denoted as $\{y_{n_x}(x)\}$, that converges. Denote $\tilde{y}$ a mapping on $X$ such that $y_{n_x}(x) \longrightarrow \tilde{y}(x)$.\medskip
	
	Since $\pi$ and $H$ are continuous, we have $ \gamma(x)= - \pi(x, \tilde{y}(x), H(x,\tilde{y}(x),u^*(x)))$.\medskip
	
	For each fixed $x\in X$, since $u_{n_x}$ are $G$-convex and $y_{n_x}(x) \in \partial^G u_{n_x}(x)$, for any $x' \in X$, we have $$u_{n_x}(x')\ge G(x', y_{n_x}(x),H(x,y_{n_x}(x),u_{n_x}(x))).$$ 
	Take limit $n_x \longrightarrow +\infty$ at both sides, we get $u^*(x')\ge G(x', \tilde{y}(x),H(x,\tilde{y}(x),u^*(x)))$, for any $x'\in X$. By definition of $G$-subdifferentiability, we have $\tilde{y}(x)\in \partial^Gu^*(x)$. \medskip





{\bf Step 3.3: } By definition of $y^*$, one has $$ -\pi(x, y^*(x), H(x,y^*(x),u^*(x)))\le   -\pi(x, \tilde{y}(x), H(x,\tilde{y}(x),u^*(x))) = \gamma(x)$$.
	
So, together with \eqref{3}, we know 
\begin{equation}\label{minimizer}
	\sup \tilde{\Pi}(u,y) \le - \int_{X}  \gamma(x) d\mu(x) \le - \int_{X}  - \pi(x, y^*(x), H(x,y^*(x),u^*(x))) d\mu(x) = \tilde{\Pi}(u^*,y^*).
\end{equation}


Since $\{u_n\}$ converges to $u^*$, and $u_n(x)\ge u_{\emptyset}(x)$ for all $n\in \N$ and $x \in X$, we have $u^*(x)\ge u_{\emptyset}(x)$ for all $x \in X$. In addition, because $u^*$ is $G$-convex and $y^*(x) \in \partial^G u^*(x)$, we know $(u^*, y^*)$ satisfies all the constraints in \eqref{Principal_new_problem}. Together with \eqref{minimizer}, we proved $(u^*,y^*)$ is a solution of the principal's program.
\end{proof}



\bigskip




%in general whether the participation constraint is binding or not.

%For continuity of solutions and the matching map, if there exist, other characteristics of the matching will be invested.






%Another interesting topic is to discover $n$-dimentional ($n\ge 2$) explicit solutions for nonlinear utilities.\medskip

\begin{thebibliography}{BCDE}
	
{	\bibitem{Armstrong96} 
	M. Armstrong, 
	Multiproduct nonlinear pricing, 
	{\em Econometrica}, {\bf 64 }(1996) 51–75.
}	
	\bibitem{Balder77} 
		E.J. Balder, 
		An extension of duality-stability relations to non-convex optimization problems, 
		{\em SIAM J. Control Optim.}, {\bf 15} (1977) 329-343.
	
	\bibitem{BaronMyerson82} 
	D.P. Baron, R.B. Myerson, 
	Regulating a monopolist with unknown costs, 
	{\em Econometrica} {\bf 50} (1982) 911–930.
	
	\bibitem{Basov05} 
	S. Basov, 
	{\em Multidimensional Screening}, 
	Springer-Verlag, Berlin, 2005.
	
	\bibitem{Carlier01} 
	G. Carlier, 
	A general existence result for the principal–agent problem with adverse selection, 
	{\em J. Math. Econom.} {\bf 35} (2001) 129–150.
	
	\bibitem{CarlierLachand-Robert01} 
	G. Carlier, T. Lachand-Robert, 
	Regularity of solutions for some variational problems subject to convexity constraint, 
	{\em Comm. Pure Appl. Math.} {\bf 54} (2001) 583–594.
	
{	\bibitem{DoleckiKurcyusz78} 
	S. Dolecki, S. Kurcyusz, 
	On $\Phi$-convexity in extremal problems, 
	{\em SIAM J. Control Optim.} {\bf 16} (1978)  277-300.
}

\bibitem{EkelandTemam76}
	I. Ekeland, R. Temam,
	 {\em Analyse convexe et probl$\acute{e}$mes variationnels},
	 Dunod (Libraire), Paris, 1976.

{
	\bibitem{ElsterNehse74} 
	K.-H. Elster, R. Nehse, 
	Zur theorie der polarfunktionale, 
	{\em Math. Operationsforsch. Stat.} {\bf 5} (1974) 3-21.
}

	\bibitem{Evans98}
	L. C. Evans, 
	{\em Partial Differential Equations},
	American Mathematical Society, Providence, Rhode Island, 1998.
	

	\bibitem{FigalliKimMcCann11} 
	A. Figalli, Y.-H. Kim, R.J. McCann, 
	When is multidimensional screening a convex program? 
	{\em J. Econom. Theory} {\bf 146} (2011) 454-478.
	
{	\bibitem{GangboMcCann96} 
	W. Gangbo, R.J. McCann, 
	The geometry of optimal transportation, 
	{\em Acta Math.} {\bf 177} (1996) 113–161.
}
	

{
	\bibitem{KutateladzeRubinov72} 
	S.S. Kutateladze, A.M. Rubinov, 
	Minkowski duality and its applications, 
	{\em Russian Math. Surveys} {\bf 27}  (1972) 137-192.
}

	
{\bibitem{MartinezLegaz05} 
	J.E. Martínez-Legaz, 
	Generalized Convex Duality and its Economic Applications, in: 
	{\em Handbook of generalized convexity and generalized monotonicity}, Springer, New York, 2005, pp. 237--292.	
}
{
	\bibitem{MaskinRiley84} 
	E. Maskin, J. Riley, 
	Monopoly with incomplete information, 
	{\em The RAND Journal of Economics} {\bf 15} (1984) 171-196.
}	
	

	
	\bibitem{McAfeeMcMillan88} 
	R.P. McAfee, J. McMillan, 
	Multidimensional incentive compatibility and mechanism design, 
	{\em J. Econom. Theory} {\bf 46} (1988) 335–354.
		
	\bibitem{McCannZhang17}
	R.J. McCann, K.S. Zhang,
	On concavity of the monopolist's problem facing consumers with nonlinear price preferences,
	To appear in {\em Comm. Pure and Applied Math.}
		
	\bibitem{Mirrlees71}  
	J.A. Mirrlees, 
	An exploration in the theory of optimum income taxation, 
	{\em Rev. Econom. Stud.} {\bf 38} (1971) 175–208.
	

	\bibitem{MonteiroPage98}  
	P.K. Monteiro, F.H. Page Jr., 
	Optimal selling mechanisms for multiproduct monopolists: incentive compatibility in the presence of budget constraints, 
	{\em J. Math. Econom.} {\bf 30} (1998) 473–502.


	\bibitem{MussaRosen78} 
	M. Mussa, S. Rosen, 
	Monopoly product and quality, 
	{\em J. Econom. Theory} {\bf 18} (1978) 301–317.

{
	\bibitem{Myerson81}
	R.B. Myerson, 
	Optimal auction design, 
	{\em  Mathematics of Operations Research} {\bf 6} (1981) 58-73.
}

	\bibitem{NoldekeSamuelson15p} 
	G. N\"oldeke, L. Samuelson, 
	The implementation duality. 
	{\em Cowles Foundation Discussion Paper}, 2015.	


	\bibitem{RochetChone98} 
	J.-C. Rochet, P. Chon$\acute{e}$, 
	Ironing sweeping and multidimensional screening, 
	{\em Econometrica} {\bf 66} (1998) 783–826.
	
	\bibitem{RochetStole03} 
	J.-C. Rochet, L.A. Stole, 
	The economics of multidimensional screening, in: 
	{\em M. Dewatripont, L.P. Hansen, S.J. Turnovsky (Eds.),  Advances in Economics and Econometrics}, Cambridge University Press, Cambridge, 2003, pp. 150-197.
	
{	\bibitem{Rubinov00a} 
	A.M. Rubinov, 
	Abstract convexity: Examples and applications, 
	{\em Optimization} {\bf 47} (2000)  1–33. 
}

{\bibitem{Singer97} 
	I. Singer, 
	{\em Abstract Convex Analysis}, Wiley-Interscience, New York, 1997.
}

	\bibitem{Spence74} 
	M. Spence, 
	Competitive and optimal responses to signals: An analysis of efficiency and distribution, 
	{\em J. Econom. Theory} {\bf 7} (1974) 296–332.

	\bibitem{Spence80} 
	M. Spence, 
	Multi-product quantity-dependent prices and profitability constraints, 
	{\em Rev. Econom. Stud.} {\bf 47} (1980) 821–841.
	
	\bibitem{Trudinger14} 
	N. S. Trudinger, 
	On the local theory of prescribed Jacobian equations, 
	{\em Discrete Contin. Dyn. Syst.} {\bf 34} (2014) 1663-1681.



	\bibitem{Wilson93} 
	R. Wilson, 
	{\em Nonlinear Pricing}, 
	Oxford University Press, Oxford, 1993.

	
	
\end{thebibliography}


\bigskip


\end{document}